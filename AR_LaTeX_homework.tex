\documentclass{article}
\usepackage[utf8]{inputenc}
\usepackage{tipa}
\usepackage{qtree}
\usepackage{phonrule}
\usepackage{amsmath}
\usepackage{graphicx}
\graphicspath{ {./pics/} }
\usepackage{wrapfig}
\usepackage{CJKutf8}

\title{LaTeX homework}
\author{Agnieszka Rachlewicz }
\date{May 2019}

\begin{document}

\maketitle

\section{Section 1}
\begin{enumerate}
  \item Neurology (brain possibilities).
  \item Phonology in general.
  \item Somnology (about dreaming in general, includes lucid dreaming).
\end{enumerate}
\section{Section 2}

% aɪm ən ˈɔːl.məʊst ˈtwɛn.ti-faɪv jɪr əʊld ˈstjuː.dənt ɒv ˈɪŋ.ɡlɪʃ fɪˈlɒl.ə.dʒi | maɪˌspeʃ.əl.aɪˈzeɪ.ʃən  ɪz ˈnætʃ.ər.əl  ˈlæŋ.ɡwɪdʒ ˈprəʊ.ses.ɪŋ  |   aɪm ɪn ðə ˈmɪdl ɒv maɪ ˌsek.ənd.dɪˈɡriː ænd maɪ ˈɪn.trəsts   ɑː  ˈlɜː.nɪŋ ˌdʒæp.ənˈiːz |  dɑːnsing   dʒæz  and   ˈfɪɡ.ə ˌskeɪ.tɪŋ wɪtʃ  ɪnˈkluːdz on ˈaɪs ˌskeɪ.tɪŋ | ˈfɪɡ.ə  ˈrəʊ.lə.skeɪ.tɪŋ and off-ice ˈek.sə.saɪzIz/
English.\newline 

\textipa{/aIm @n "O:l.m@Ust "twEn.ti faIv jI@ @Uld "stju:.d@nt @v "IN.glIS fI"l6l.@\texttoptiebar{\textdyoghlig}i || maI ""spES.@l.aI"zeI.S@n Iz "n\ae \texttoptiebar{\textteshlig}.@r.@l "l\ae \ng.gwI\texttoptiebar{\textdyoghlig} \space "pr@U.sEs.I\ng \space || aIm In \dh @ "mId.l @v maI ""sEk.@nd.dI"gri: | @nd maI "In.tr@sts \textscripta: | "l3:.nI\ng \space ""\texttoptiebar{\textdyoghlig}\ae p.@n"i:z | d\textscripta:nsI\ng \space \texttoptiebar{\textdyoghlig}\ae z | @nd  
"fIg.@ ""skeI.tI\ng \space wI\texttoptiebar{\textteshlig} In"klu:\texttoptiebar{dz} | 6n aIs ""skeI.tI\ng \space | "fIg.@ "r@U.l@.skeI.tI\ng \space | @nd 6f aIs "ek.s@.saIzIz/}
\newline \newline
Polski.
% Jestem dwudziestoczteroletnią studentką filologii angielskiej, na specjalizacji przetwarzanie języka naturalnego. 
%Moimi zainteresowaniami są nauka języka japońskiego, tańczenie dżezu oraz jazda figurowa, która obejmuje jazdę na łyżwach, jazdę na łyżworolkach figurowych oraz imitacje czyli ćwiczenia poza lodem.
\newline
\newline
\textipa{/jEstEm dvu\textdctzlig est\textopeno \texttoptiebar{\textteshlig}ter\textopeno let\textltailn  \~\textopeno \space studentk\~\textopeno \space fil\textopeno l\textopeno g\textsuperscript{j}i \~a\ng g\textsuperscript{j}Elsk\textsuperscript{j}ej | na spEcjal\textsuperscript{j}izac\textsuperscript{j}i pSEtfa\textyogh an\textsuperscript{j}e j\~E\~wz\textbari ka naturalnEg\textopeno \space || m\textopeno \textsuperscript{j}imi zaintErEs\textopeno van\textsuperscript{j}ami s\~\textopeno \space | nauka j\~Ez\textbari ka naturalnEg\textopeno \space | t\~a\~j\texttoptiebar{\textteshlig}En\textsuperscript{j}E d\texttoptiebar{\textyogh}Ezu \textopeno raz jazda figur\textopeno va | ktura obEjmujE jazdE na w\textbari\textyogh vax | jazdE na w\textbari\textyogh v\textopeno r\textopeno lkax figur\textopeno v1x | \textopeno raz imitacjE | \texttoptiebar{\textteshlig}1li t\texttoptiebar{\textctc}fi\textteshlig En\textsuperscript{j}a p\textopeno za l\textopeno dEm/}

\section{Section 3}
%"We will go home at the end of the day." 

\Tree
[.CP [.C\1 [ (decl.) ].C [.TP [.NP [.N\1 [ We ].N
]] 
[ [.T\1 [ [ will ].T  ] 
[.VP  
[.V\1 
[ [ [ go ].V \qroof{home}.NP ] ].V\1 
[.PP
[.P\1 [ at ].P
[.DP
[.D\1 
[ the ].D [.NP [.N\1 [ end ].N 
[.PP [.P\1 [ of ].P 
[.DP [.D\1 [ the ].D \qroof{day}.NP 
]]]]]]]]]]]]]].I\1 ]]]
\newline
\newline
\newline

\section{Section 4}

Rule 1: Spirantization
\newline
\newline
\phonc{
\phonfeat[l]{+stop \\
--voice}}{
\phonfeat[l]{--stop \\
+voice\\+fricative}}{
\phonfeat[l]{+vowel}
\phold
\phold
\phonfeat[l]{+vowel}
}
\newline\newline

\textit{A voiceless stop is realized as the corresponding voiced fricative when
surrounded by vowels.}
\newline
\newline
Rule 2: Vowel Nasalization
\newline

\phonc{
\phonfeat[l]{+vowel}
}{
\phonfeat[l]{+nasal}
}{
\phold
\phold
\phonfeat[l]{+nasal}}
\newline
\newline
\textit{A vowel is realized as nasalized when it precedes a nasal consonant.}
\newline
\newline
Rule 3: Kongo Palatalization
\newline

\phonc{
\phonfeat[l]{+coronal\\
--sonorant}
}{
\phonfeat[l]{+delayed release\\
--anterior\\
+distributed\\
+strident}
}{
\phold
\phold
\space i}

\section{Section 5}
This linguistic equation is taken from András Kornai's \textit{Mathematical linguistics} (2008) page 36 (47 in the pdf file).
\newline
\newline
3.3
\begin{align*}
f(n+1,k+1) &= \sum_{i=1}^{k+1} f(n,i)2^{k+1-i} + f(n,k+1)
\end{align*}

\section{Section 6}
%The line \usepackage{CJKutf8} imports CJKutf8 which enables utf8 encoding for Chinese, Japanese and Korean fonts.
\includegraphics[scale=0.1]{pics/Yuzu1.png}
\begin{CJK}{UTF8}{min}
羽生・結弦(1994年12月7日 --  )
\end{CJK} 
\textipa{/han\textsuperscript{j}\textturnm\textturnm\space j\textturnm\textdzlig\textturnm\textfishhookr\textturnm/} 
\newline

\begin{CJK}{UTF8}{min}
羽生・結弦は、宮城県仙台市泉区出身のフィギュアスケート選手(男子シングル)。主要な戦績として、2014年ソチオリンピック・2018年平昌オリンピック2大会連続優勝(男子シングル種目で66年ぶりの2連覇)。
\begin{wrapfigure}{r}{0.45\textwidth} 
    \centering
    \includegraphics[width=0.45\textwidth]{pics/Sochi_2014.jpg}
\end{wrapfigure}
つまり、結弦さんはオリンピックの金メダルを二枚勝ちました。
\newpage
\newline

最後の写真は今年のファンタジーオンアイスの出演から。
羽生・結弦さんは私と同じ年に生まれました。彼は私を動かします。
\end{CJK}
\newline
\begin{wrapfigure}{r}{0.35\textwidth} 
    \centering
    \includegraphics[width=0.35\textwidth]{pics/Yuzu2.png}
\end{wrapfigure}
Yuzuru Hanyu is a figure skater (in single men category) from Izumiku, Sendai, Miyagi Prefecture, Japan. His major achievements(results) are, among others, (after a 66-year break for Japan) 2 consecutive victories in the 2014 Sochi Olympic Winter Games and the 2018 PyeongChang Olympic Winter Games. 
\newline
In other words, Yuzuru won (consecutively) 2 Olympic gold medals. \newline

The last picture is from the 2019 Fantasy on Ice performance. Yuzuru Hanyu was born the same year as I. He inspires me.
\end{document}